\documentclass[12pt]{article}
\usepackage{amsfonts,amssymb,amsmath}
%\documentstyle[12pt,amsfonts]{article}
%\documentstyle{article}

\setlength{\topmargin}{-.5in}
\setlength{\oddsidemargin}{0 in}
\setlength{\evensidemargin}{0 in}
\setlength{\textwidth}{6.5truein}
\setlength{\textheight}{8.5truein}
\setcounter{MaxMatrixCols}{15}
%\input ../basicmath/basicmathmac.tex
%
%\input ../adgeomcs/lamacb.tex
\input ../adgeomcs/mac-new.tex
\input ../adgeomcs/mathmac-v2.tex
%\input ../adgeomcs/mac.tex
%\input ../adgeomcs/mathmac.tex

\def\fseq#1#2{(#1_{#2})_{#2\geq 1}}
\def\fsseq#1#2#3{(#1_{#3(#2)})_{#2\geq 1}}
\def\qleq{\sqsubseteq}

%
\begin{document}
\vspace {0.25cm} \noindent
{\bf Problem B1}
(4)
The two rows given can be obtained by applying the following row operations respectively: 
$-2A_1+A_3$ and $0.5(3A_1 - A_3)$. Denoting those two rows as $B_1$ and $B_2$, 
$A_2$ can be expressed as $2B_1 + 3B_2$. In fact, every subsequent row $A_n$ can be 
expressed as $nB_1 + (n+1)B_2$. 

\vspace {0.25cm} \noindent
{\bf Problem B3}
(3)

(4)
There are no normal magic squares for $n=2$. By the property of magic squares, given 
$A = \begin{pmatrix}
  a_1 & a_2\cr a_3 & a_4
\end{pmatrix}$
it must be the case that $a_1 + a_2 = a_1 + a_3$. However, this is equivalent to $a_2 = a_3$, 
which violates the constraint that all the elements must be distinct.
\begin{align*}
  1 + ... + 3^2 &= \frac{2x_1 + 2x_2 - x_3 + 2x_1 - x_2 + 2x_3 - x_1 + 2x_2 + 2x_3 
  - 2x_1 + x_2 + 4x_3 + x_1 + x_2 + x_3 + 4x_1 + x_2 - 2x_3}{3} + x_1 + x_2 + x_3\\
  45 &= 3x_1 + 3 x_2 + 3x_3\\
  15 &= x_1 + x_2 + x_3
\end{align*}
\end{document}
